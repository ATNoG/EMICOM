% Created 2009-12-11 Fri 18:50
\documentclass[11pt]{article}
\usepackage[utf8]{inputenc}
\usepackage[T1]{fontenc}
\usepackage{graphicx}
\usepackage{longtable}
\usepackage{hyperref}


\title{ \begin{Huge} \textbf{- ODTONE -} \end{Huge} \\ \vspace*{1cm}
- \textbf{O}pen \textbf{D}ot \textbf{T}wenty \textbf{One} - \\ \vspace*{3cm}
Installation Manual \& User Guide \vspace*{3cm}}
\author{Simao Reis,
% <\texttt{sreis@av.it.pt}>
% \\
Rui Costa
% <\texttt{rpfcosta@av.it.pt}>
\vspace*{0.5cm}}
\date{\today}

\begin{document}

\maketitle
\newpage

\setcounter{tocdepth}{3}
\tableofcontents
\vspace*{1cm}
\newpage

\section{\texttt{ODTONE} Installation}
\label{sec-1}

As \texttt{ODTONE} is developed in order to work in several platforms, we therefore
won't provide detailed tutorials for each mainstream operating system. We will rather
present the main guidelines for installing \texttt{ODTONE} and any relevant notes on
specific platforms that might prove an obstacle.
% \subsection{Installation Guide}
% \label{sec-1.1}

\subsection{Required Dependencies}
% \label{sec-1.1.1}
\label{sec-1.1}

In order to correctly use \texttt{ODTONE} you must first make sure you have the
required dependencies installed.

\begin{itemize}
\item \textit{Boost} Developer libraries (v1.37 or higher)
	\subitem	If you do not already have them in your system, you can go to
			\url{http://www.boost.org/} and download a more recent version
			of the \textit{Boost} libraries and manually compile them.

% or, if you are running \texttt{Ubuntu 9.10}
% 			(or higher) just run the following command on
% 			the shell:
%
%    	\texttt{\$ apt-get install libboost-dev}


\item \textit{SQLite} Developer library (v3.5.9 or higher)
	\subitem	If you do not already have them in your system, you can go to
			\url{http://www.sqlite.org/} and download a more recent version.


% 	\subitem	You can install this library by simply running the following
% 			command on the shell:
%
% 	\texttt{ \$ apt-get install libsqlite3-dev}


\end{itemize}

\subsection{Getting \texttt{ODTONE}}
\label{sec-1.2}


\begin{itemize}
\item	In a \textit{tarball}
	\subitem	You can get the \texttt{ODTONE} source code \textit{tarball}
from our project page, at \url{http://helios.av.it.pt/projects/odtone/files}. Then
just unpack it and you are ready to compile and install \texttt{ODTONE}.

\item	From the \textit{git} repository
You can \textit{clone} the \textit{git} repository, from \href{http://helios.av.it.pt/projects/odtone/git/}{http://helios.av.it.pt/projects/odtone/git/}
therefore creating your own local copy.



\end{itemize}




% Unpack the tarball archive, cd into the directory, then
% compiling the project is as easy as: \\
% \\
%    \$ tar xzf odtone-<version>.tar.gz \\
%    \$ cd odtone-<version>/ \\
%    \$ ./configure \\
%    \$ make \\

% You can check out the latest version from the \texttt{ODTONE} \textit{Git} repository by typing the
% following command in your console:
%
%     \$ git clone \href{http://helios.av.it.pt/projects/odtone/git/}{http://helios.av.it.pt/projects/odtone/git/} <local\_odtone\_repository\_folder>
%
% This will create the an odtone directory with the source code. To compile the project: \\
% \\
%    \$ cd odtone/ \\
%    \$ ./configure \\
%    \$ make \\


\subsection{Compilling and installing \texttt{ODTONE}}
\label{sec-1.3}

Once you have obtained the \texttt{ODTONE} source code and have assured you have the
right dependencies are installed in all machines you intend to use \texttt{ODTONE},
then you are ready to compile and install.

\begin{itemize}
\item	Windows
	\subitem	Using \textit{Visual Studio 2008}, choose the option to open an
existing project. Browse to the folder \textit{vc90/} on your \texttt{ODTONE} folder and
select and open the \texttt{ODTONE} project. Then just build the project.

\textbf{NOTE} that this is still an \textit{alpha} version, therefore in future releases \textit{Windows}
support will be improved, in this manual.

\item	Linux/Unix
	\subitem	You just need to run the \textit{``./autogen.sh''} script and
    then run \texttt{./configure} and finally \texttt{make} to compile the code. Install is not yet required, and therefore
all applications are local and can be found under the \textit{src/} folder.


\end{itemize}


















% ====================================================================================================================

\newpage
\section{\texttt{ODTONE} User Guide}
\label{sec-2}

   For now the \texttt{MIHF}s capabilities are read from configuration
   files. Future versions will inquire the available \texttt{Link SAP}s for the
   required information.


\subsection{Local Demo}
\label{sec-2.1}

The local demo consists in a simple experiment to demonstrate simple message
exchange between a \texttt{MIHF}, a \texttt{MIH\_User} and a \texttt{Link\_SAP}.
It allows you to see how evenst are generated by the \texttt{Link\_SAP} and
reported to a \texttt{MIH\_User} that has subscribed to these events.

\subsubsection{Configuration}
\label{sec-2.1.1}

   To run the local demo you will need to edit the default
   configuration file \emph{odtone.conf} that is stored in the \textit{src/}
   directory. The file is well documented and you will need to edit
   the \emph{link\_addr\_list} entrance and add the MAC address of your
   computer's network cards. \\

%
%    Run the following command to find the required information. \\
% \\
%    \$ ifconfig | grep HW


\subsubsection{Running the Demo}
\label{sec-2.1.2}


   After editing the configuration file, to run this \texttt{ODTONE} demo the
   best way is to open 3 terminals. On one terminal start \texttt{ODTONE},
   and on the next terminal start the \texttt{MIH\_User}. \\
%
% by
%    cd'ing into src/ directory and starting the odtone executable. \\
% \\
%     \$ cd odtone/src \\
%     \$ ./odtone \\
%
%     On another terminal start the \texttt{MIH\_User}\\
% \\
%     \$ cd odtone/src \\
%     \$ ./mih\_usr \\

    If all went well the \texttt{MIH\_User} has requested an \texttt{Capability\_Discover}
    to the local \texttt{MIHF} and printed out some information of the
    interfaces you previously configured. \\

    Now it's time to start the \texttt{Link\_SAP}, on your third terminal, and
    send some Link indications to the \texttt{MIHF} and check that the \texttt{MIH\_User}
    received notifications . \\
% \\
%     \$ cd odtone/src/link\_sap \\
%     \$ ./link\_sap \\
%

    So that the \texttt{Link\_SAP} detects some events you should now proceed to
    disconnect, disable or shutdown you network cable/interface or your wireless card according
    to your configuration file (\textit{odtone.conf}).

%     If your Linux distribution uses Network Manager just right click on
%     the applet's icon, and then enable/disable networking and/or
%     wireless.
%
%     If you don't have Network Manager just use the \emph{ifconfig}
%     utility on you terminal. \\
% \\
%     To disconnect your ethernet card: \\
% \\
%     \$ sudo ifconfig eth0 down \\
% \\
%     To bring it back up: \\
% \\
%     \$ sudo ifconfig eth0 up
% \\


\subsection{Remote Demo}
\label{sec-2.2}

The remote demo consists in a local \texttt{MIH\_User} obtaining notifications of events
that are happening on a \texttt{Link\_SAP} remotely located on another machine.
You are required to have two machines for this experiment, the first machine hosting an \texttt{ODTONE}
instance (\texttt{mihf1}) and the \texttt{MIH\_User}, while the second will host a second \texttt{ODTONE}
instance \texttt{mihf2}) and the \texttt{Link\_SAP}.



\subsubsection{Configuration}
\label{sec-2.2.1}

    As was said earlier, automatic remote peer \texttt{MIHF} discovery is not
    yet implemented, so you need to add the remote \texttt{MIHF}s IP address and port
    number to the configuration file (\emph{odtone.conf}).

    Edit the file and add an entry to \emph{peers} in the form:\\
    \\
    \texttt{<mihf\_id> <ip> <port> \\}
    \\
    \texttt{<mihf\_id>} is the identifier of the remote \texttt{MIHF}, \texttt{<ip>}
    and \texttt{<port>} are self explanatory.\\
    \\
    \textbf{NOTE} that you need to edit the configuration file on both
    machines. \\
    \\
%
%     In this example scenario we assume that mihf1 has one user, and
%     the mihf2 has one link sap. And we want the user to subscribe to
%     the remote MIHFs link events.



\subsubsection{\texttt{mihf1} configuration}
\label{sec-2.2.2}


    On the machine with the \texttt{mihf1} the config file can look like:\\
    \begin{verbatim}
    [mihf]
    id = mihf1
    local_port = 1025
    remote_port = 4551
    peers = mihf2 <mihf2_IP_address> 4551
    users = user 1234
    \end{verbatim}


\subsubsection{\texttt{mihf2} configuration}
\label{sec-2.2.3}

    The configuration file of \texttt{mihf2} would be: \\
    \begin{verbatim}
    [mihf]
    id = mihf2
    local_port = 1025
    remote_port = 4551
    peers = mihf1 <mihf1_IP_address> 4551
    links = link 1235
    link_addr_list = 802_11 <mac_address>,ethernet <mac_address>
    event_list = link_detected, link_up, link_down,
    link_parameters_report, link_going_down,
    link_handover_imminent, link_handover_complete
    \end{verbatim}

    \textbf{NOTE} that you need to setup the MAC addresses of the
    \texttt{mihf2} interfaces. \\

\subsubsection{Running the Demo}
\label{sec-2.2.4}

\begin{itemize}

 \item  On the machine hosting \texttt{mihf1}, start a terminal and run \texttt{ODTONE}.
	Then, on another terminal start the \texttt{MIH\_User} adding the parameter
	\texttt{\textendash\textendash dest mihf2}.
%
%
%
%
%
%     To start the demo:
%     \begin{verbatim}
%      $ cd odtone/src
%      $ ./odtone
%      \end{verbatim}
%      In another terminal on the same machine: \\
%      \begin{verbatim}
%      $ cd odtone/src
%      $ ./mih_usr --dest mihf2
%      \end{verbatim}
     The \emph{\textendash\textendash dest} option tells the \texttt{MIH\_User} to set the \texttt{802.21}
     destination field of the frame to \texttt{mihf2}.

 \item  On the machine hosting \texttt{mihf2}, start a terminal and run \texttt{ODTONE}.
	Then, on another terminal start the \texttt{Link\_SAP}. So that the \texttt{Link\_SAP}
	detects some events you should now proceed to disconnect, disable or shutdown your
	network cable/interface or you wireless card according to your configuration file (\textit{odtone.conf}).

%
%
%
%
%
%     To start the demo:
%     \begin{verbatim}
%     $ cd odtone/src
%     $ ./odtone
%     \end{verbatim}
%      In another terminal on the same machine: \\
%      \begin{verbatim}
%      $ cd odtone/src/link_sap
%      $ ./link_sap
%      \end{verbatim}

\end{itemize}


\end{document}
